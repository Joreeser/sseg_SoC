\documentclass{article}
\usepackage{graphicx} % new way of doing eps files
\usepackage{listings} % nice code layout
\usepackage[usenames]{color} % color
\usepackage{hyperref}
\definecolor{listinggray}{gray}{0.9}
\definecolor{graphgray}{gray}{0.7}
\definecolor{ans}{rgb}{1,0,0}
\definecolor{blue}{rgb}{0,0,1}
% \Verilog{title}{label}{file}
%\newcommand{\Verilog}[3]{
% \lstset{language=Verilog}
% \lstset{backgroundcolor=\color{listinggray},rulecolor=\color{blue}}
% \lstset{linewidth=\textwidth}
% \lstset{commentstyle=\textit, stringstyle=\upshape,showspaces=false}
% \lstset{frame=tb}
% \lstinputlisting[caption={#1},label={#2}]{#3}
%}
\newcommand{\Cpp}[3]{
	\lstset{language=C++}
	\lstset{backgroundcolor=\color{listinggray},rulecolor=\color{blue}}
	\lstset{linewidth=\textwidth}
	\lstset{commentstyle=\textit, stringstyle=\upshape,showspaces=false}
	\lstset{frame=tb}
	\lstinputlisting[caption={#1},label={#2}]{#3}
}


\author{Jordan Reeser}
\title{SSEG SoC}

\begin{document}
\maketitle

\section{Executive Summary}
In this lab, students were responsible for implementing the seven-segment (sseg) display into the Vanilla SoC. This was was accomplished by creating a generic version of the sseg implementation done in the scrolling display, making a wrapper for the sseg module, and connecting this module to the FPRO bridge. In order to do this, Xilinx SDK was used along with Vivado to create the ELF file and program the board. For the sake of simplicity, the display value of the 8-digit sseg is set to 0xFFFFFFFF to show that the values can be be changed from the default values and be properly displayed. The following link has the GitHub repository for this lab: \url{https://github.com/Joreeser/Scrolling_Display}

\section{Creating SSEG Module}
The sseg display used in this lab was adapted from the sseg display used in the Scrolling Display lab. The BRAM is no longer included, and the refresh rate is now programmable. The display as also static instead of scrolling accross the 8-digits. 

\section{Implementing SSEG into FPRO Bus}
The sseg module that was created then had to be implemented into the FPRO bus. This was acccomplished by first creating a wrapper for the sseg module. The display value and refresh value are held in separate registers and the default bridge signals of chip select, read enable, write enable, address, read data, and write data are also connected. This wrapper was then plugged in to the bridge by adding it to the mmio\_sys\_vanilla.sv file and adding the sseg and an signals to that file along with mcs\_top\_vanilla.sv file that instantiates mmio\_sys\_vanilla.sv.

\section{Programming SSEG in SDK}
In order to control the functionality of the newly implemented sseg, a workspace was created in SDK for the SoC to create the ELF file. The following files had to be created/adapted to accomplish this:

\begin{enumerate}
	\item main\_vanilla\_test.cpp
	\item sseg\_core.cpp
	\item sseg\_core.h
\end{enumerate} 

\subsection{main\_vanilla\_test.cpp}
This is the test code file that sruns the functionality of the SoC. To implement the the sseg, a function had to be made that sets the display value and refresh rate of the sseg. For the sake of simplicity, the display value is set to 0xFFFFFFFF and the refresh rate is set to 3 to have a solid display. It is important to note that at bootup, the sseg displays the default value, 0, at the default refresh rate, 50, or a short period of time before using the new values. 

\subsection{sseg\_core.cpp}
this file creates the sseg core and its functions. In the core definition, the default refresh rate is set to 50 and the default display value is set to 0. The functions for the sseg core are set\_display\_data, which simply sets the hex values to be displayed, and set\_refresh\_rate, which sets the sseg display refresh rate in ms. 

\subsection{sseg\_core.h}
This file is the header file for the sseg core. It sets up the class and enumerates the data register (DATA\_REG) and refresh register (REFRESH\_REG) to 0 and 1 respectively. It also has the function primitives for the sseg functions. This file also sets up the base\_addr variable as private to the class.


\section{Code Appendix}
\Cpp{C++ code for implementing the SoC functionality with the sseg.}{code:main_vanilla_test}{../nexys4_sv_vanilla/sseg_test/main_vanilla_test.cpp}

\Cpp{C++ code for implementing the sseg core with functions to set display data and refresh rate.}{code:sseg_core_c}{../nexys4_sv_vanilla/sseg_test/sseg_core.cpp}

\Cpp{C++ header file for the sseg_core.}{code:sseg_core_h}{../nexys4_sv_vanilla/sseg_test/sseg_core.h}
\end{document} 